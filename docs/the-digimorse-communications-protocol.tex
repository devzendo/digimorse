\documentclass{tufte-handout}

\title{The digimorse Communications Protocol}

\author{Matt Gumbley, M{\emptyset}CUV}

%\date{6 September 2021} % without \date command, current date is supplied

%\geometry{showframe} % display margins for debugging page layout

\usepackage{graphicx} % allow embedded images
\setkeys{Gin}{width=\linewidth,totalheight=\textheight,keepaspectratio}
\graphicspath{{graphics/}} % set of paths to search for images
\usepackage{amsmath}  % extended mathematics
\usepackage{booktabs} % book-quality tables
\usepackage{units}    % non-stacked fractions and better unit spacing
\usepackage{multicol} % multiple column layout facilities
\usepackage{lipsum}   % filler text
\usepackage{fancyvrb} % extended verbatim environments
\usepackage{hyperref} % URLs
\fvset{fontsize=\normalsize}% default font size for fancy-verbatim environments

% Standardize command font styles and environments
\newcommand{\doccmd}[1]{\texttt{\textbackslash#1}}% command name -- adds backslash automatically
\newcommand{\docopt}[1]{\ensuremath{\langle}\textrm{\textit{#1}}\ensuremath{\rangle}}% optional command argument
\newcommand{\docarg}[1]{\textrm{\textit{#1}}}% (required) command argument
\newcommand{\docenv}[1]{\textsf{#1}}% environment name
\newcommand{\docpkg}[1]{\texttt{#1}}% package name
\newcommand{\doccls}[1]{\texttt{#1}}% document class name
\newcommand{\docclsopt}[1]{\texttt{#1}}% document class option name
\newenvironment{docspec}{\begin{quote}\noindent}{\end{quote}}% command specification environment

\begin{document}

    \maketitle% this prints the handout title, author, and date

    \begin{abstract}
        \noindent
        This document describes the rationale for, and the design of the Digimorse communications protocol.
        It describes choices made at all stages of the “transceiver”.
    \end{abstract}

%\printclassoptions


Just to get the build of this working in IntelliJ, I'll have to cite something, and I'll be referring
to Gallager's initial paper on LDPC~\cite{Gallager1962} quite a bit..






\section{Project Website}

The website for the digimorse protocol and transceiver software is located at
\url{https://devzendo.github.io/digimorse}. On the website, you'll find
links to our \smallcaps{git} repository, mailing lists, bug tracker, and documentation.

\bibliography{the-digimorse-communications-protocol}
\bibliographystyle{plainnat}



\end{document}
